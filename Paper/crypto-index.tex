\documentclass[11pt]{article}

\usepackage{amsmath}
\usepackage{amssymb}
\usepackage{bbm}
\usepackage{natbib}
\usepackage{color}
\usepackage{caption}
\usepackage{rotating}
\usepackage{hyperref}

\title{Lykke Index 20}
\author{Onno Kleen \and Christopher Zuber}
\begin{document}
\maketitle
\section{Index}

Some Text.\footnote{Code and .tex-files can be found at \href{https://github.com/onnokleen/crypto-index}{https://github.com/onnokleen/crypto-index}}

\subsection{Definition}

\begin{itemize}
  \item Price $p_{i,t}$: price of asset $i$ at time $t$
  \item Quantity $q_{i,t}$: overall number of shares/items per asset $i$ at time $t$
\end{itemize}
\subsection{Questions to address}
\begin{itemize}
  \item Why 20 currencies? 19-09-2017 14:41 20th market capitalisation (STEEM) is only \$286.382.955	and 24 hour trading volume of \$686.
  \item ``Dead coins'' a problem?
  \item If there is a split (like Bitcoin), new currency is part of Lykke 20 but is part of constituents-check at the end of the week.
  \item Basis: 1000 Punkte?
  \item How to get market capitalization of public float?
  \item Maximal weight maybe 20\%? DAX: Maximum weight 10\%.
\end{itemize}

\subsection{How does it work in other indices}

\begin{itemize}
  \item DAX: Weighting based on market capitalization of public float (bedeutet Streubesitz, keine Aktien von Langzeitanlegern  wie Familie Porsche/Quant).
\end{itemize}

\subsection{Features}

\begin{itemize}
  \item 
  \item Constituent changes each week. Maybe Friday? Maybe based on trade volume in last 7 days? Good against ``dead coins''.
\end{itemize}


\section{Example}

Data is from \href{https://www.kaggle.com/sudalairajkumar/cryptocurrencypricehistory}{https://www.kaggle.com/sudalairajkumar/cryptocurrencypricehistory}

or CoinCap.io via Rest API

Something nice to illustrate:
\begin{itemize}
  \item Volatility in August (Bitcoin-split versus July). Show new composition after split.
\end{itemize}

\end{document}
